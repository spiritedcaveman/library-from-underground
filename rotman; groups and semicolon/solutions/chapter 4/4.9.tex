\documentclass{article}
\usepackage{amsmath, amssymb}

\begin{document}

For Exercise 4.9 (i) and (ii):

Let $x \in X$. Since $H$ acts transitively on $X$, for any $g \in G$, we can find $h \in H$ such that:
$$
h \cdot x = g \cdot x \quad \Rightarrow \quad h^{-1}g \cdot x = x \quad \Rightarrow \quad h^{-1}g \in G_x.
$$
So $g = h \cdot (h^{-1}g) \in H G_x$. Hence:
$$
G = H G_x.
$$

The Frattini argument follows directly from the general fact that \textbf{if a subgroup $H \le G$ acts transitively on a finite $G$-set $X$}, then $G = H G_x$ for any $x \in X$. In the Frattini setting, \textbf{$G$ acts on the set of Sylow $p$-subgroups of a normal subgroup $K$ by conjugation, and $K$ acts transitively on this set by Sylow’s theorem.} Fixing a Sylow $p$-subgroup $P \le K$, its stabilizer under the action is $N_G(P)$. Thus, applying the transitivity result yields $G = K N_G(P)$, which is precisely the Frattini argument.

\end{document}